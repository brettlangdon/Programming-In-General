Control statements are almost exactly as they sound, statements that control our programs.
Well, they control the flow of our code.
With control statements we can change the course of our programs based on various conditions.

\subsection{If Statements}
If statement allow us to execute a given block of code based on a given condition.
There are three main parts to an if statement \pigVal{if}, the \pigVal{conditional} and a \pigVal{code block}.

\begin{lstlisting}[caption={If Statement}]
name = ``brett''

if( name == ``brett'' )
    print ``Name is brett''
\end{lstlisting}

In this simple example the code block \pigOut{print ``Name is brett''} will only execute if the conditional \pigOut{name == ``brett''} is true.
So the output of this code will be \pigOut{Name is brett}.

\begin{lstlisting}[caption={False If Statement}]
name = ``brett''

if( name == ``john'' )
    print ``Name is john''
\end{lstlisting}

In this example there will be no output, this is because the conditional \pigOut{name == ``john''} equates to false so the code block \pigOut{print ``Name is john''} will never get executed.

\subsection{If-Else Statements}
If statements are great and help us execute portions of our code based on the values of other variables, including based on input from users.
But what if the condition of the if statement equates to false?
With if statements we can append an else statement and a block of code to the end of an if statement that will get called if the if statement is false. 

\begin{lstlisting}[caption={If-Else Statement}]
name = ``brett''

if name == ``john'':
    print ``Name is john''
else:
    print ``Name is not john''
\end{lstlisting}

The output of this program will be \pigOut{Name is not john}.
When the program hits the if statement it evaluates the conditional \pigVar{name == ``john''} which evaluates to \pigVal{false}.
Normally the program will continue on its way but since we provided an else statement that gets executed instead.
An If-Else statement allows us to program ``if this then do this, otherwise do this.''

\subsection{Else If Statements}
Ok... wait, we just did If-Else statements not we are doing Else if statements?
Yes, but they are different I swear!
An If-Else statement allows us to execute code regardless of whether a conditional is true or false but with an else if statement we can provide multiple conditionals to an if statements.

\begin{lstlisting}[caption={Else If Statement}]
name = ``brett''

if name == ``john'':
    print ``Name is john''
else if name == ``brett''
    print ``Name is brett''
\end{lstlisting}

See? I told you it was different.
So the output of this program is \pigOut{Name is brett} and this is because when the program gets to the if statement and evaluates it as false, it then continues down the list of conditionals.
This works similar to how the else statement before worked, but this time we are giving the if statement multiple conditionals to check.
We can expand this example by adding more else if statements.

\begin{lstlisting}[caption={Else If Statement 2}]
name = ``brett''

if name == ``john'':
    print ``Name is john''
else if name == ``brett''
    print ``Name is brett''
else if name == ``barbara'':
    print ``Name is barbara''
\end{lstlisting}

Just like the first example this program will output \pigOut{Name is brett}.
This is because when the program gets to \pigVar{name == ``john''} it evaluates to false causing the program to skip to the next conditional \pigVar{name ==''brett''}, which then evaluates to true causing the code block given to execute.
The last conditional \pigVar{name == ``barbara''} will then be skipped and the program will continue past the if statement.
\par

Now... what is we add an else statement to the end of this?
With an if statement we could append an else statement to the end telling it what to do if the conditional failed.
With an else if statement we can also append an else statement to the end telling it what to do if all of the conditionals fail.

\begin{lstlisting}[caption={Else If Else Statement}]
name = ``brett''

if name == ``john'':
    print ``Name is john''
else if name == ``barbara'':
    print ``Name is barbara''
else:
   print ``Well, I`m not sure what your name is''
\end{lstlisting}

This program will output \pigOut{Well, I'm not sure what your name is} because both conditionals, \pigVar{name == ``john''} and \pigVar{name == ``barbara''}, evaluate to false causing the if statement to continue on its merry way.

\subsection{Switch Statements}
A Switch statement is similar to a grouping of If, Else If and Else statements but where the conditional is always a direct comparison to a value.
Switch statements are useful when you have a set number of values to compare a variable against.
For example, the following If statements are a perfect candidate for a switch statement.

\begin{lstlisting}[caption={Switch Statement Candidate}]
name = ``brett''

if name == ``john'':
    print ``name is john''
else if name == ``barbara'':
    print ``name is barbara''
else if name == ``eugene'':
    print ``name is eugene''
else if name == ``brett'':
    print ``name is brett''
else:
    print ``not sure what your name is''
\end{lstlisting}

With a Switch statement it can be rewritten as.

\begin{lstlisting}[caption={Switch Statement Example}]
name = ``brett''

switch name:
    case ``john'':
        print ``name is john''
        break
    case ``barbara'':
        print ``name is barbara''
        break
    case ``eugene'':
        print ``name is eugene''
        break
    case ``brett'':
        print ``name is brett''
        break
    default:
        print ``not sure what your name is''
        break
\end{lstlisting}

Both of these programs work in a similar manner, take a variable and do a direct comparison to a set of values until a match is made or else use a default action.
As well they will both output the same \pigOut{name is brett}.
Think of a Switch statement as a set of If, Else If, Else statements where the conditionals are always a single \pigVar{==}.
\par

A switch statement introduces a few new keywords, the switch followed by the variable name we wish to compare against.
Then we can have as many case statements following, each with the value that we wish to compare our variable against.
The only other weird part is that we are also introducing the break statement, which is required to terminate each case statement code block.
What the break statement says to do is ``break'' away from the entire switch statement.
As an excersise, try removing all of the break statements from the above example and run it again, what changed?
\par

We have mainly been comparing string variables against string values but you can also use Switch statements to compare numbers as well.

\begin{lstlisting}[caption={Switch Statement Numbers Example}]
age = 22

switch age:
    case 20:
        print ``not old enough to drink''
        break
    case 21:
        print ``congratulations, do not over do it''
        break
    case 22:
        print ``you`ve been doing this awhile''
        break
\end{lstlisting}

As you can see, we can also compare our number variable against number values.
In this example we also have left out the default case, this case is optional, similar to the else statement.

\subsection{For Loops}

\subsection{While Loops}

\subsection{Do-While Loops}

\subsection{Break Statements}

\subsection{Continue Statements}

\subsection{Conclusion}
