Variables act as aliases to the values that we want them to represent and they allow us to access and manipulate the
values that we assign to them.
For example we could use the variable \pigVar{name} to represent the value \pigVal{"Brett Langdon"}.
We do this with programming so that we can then access the value \pigVal{"Brett Langdon"} with a shorter representation, \pigVar{name}.


\subsection{Declaration}

To start with variables we need to declare their existence.
By declaring a variable we are saying to the program, here is our alias and here is the value that we want it to represent.

\begin{lstlisting}[caption={Variable Declaration}]
a = 10
print a
\end{lstlisting}

In the above example we are saying that we want to store the integer value \pigVal{10} into the variable \pigVar{a}.
We can then use the variable \pigVar{a} to access the value \pigVal{10}.
This program will output \pigOut{10} rather than \pigOut{a}.
\par

When we declare variables we are telling the programming language to allocate some space in your computers memory
in order to store the value that you need it to.
The amount of space needed to store each variable depends based on your specific language being used and which data
type is being used to store the value.



\subsection{Data Types}

Programming languages support different types of data types or different types of values that they can represent in variables.
Some programming languages use multiple different types of values but most of them support the basic types: string, integer (multiple kinds)
and boolean (true or false).

\begin{lstlisting}[caption={Data Types}]
string = "Brett"
integer = 10
boolean = false
\end{lstlisting}

Please keep in mind that each programming language supports different data types and you should research those types to better
understand variables in that language.
As well some programming languages that are strickly typed which requires us to define the data type of the
variable on declaration (unlike our sudo language).


\subsection{Operators}
Operators are symbols we can use to manipulate variables.
There are a few different types of operators, Arithmetic, Comparison and Logical operators.
A quick list of operators include:

\begin{enumerate}
\item Addition +
\item Subtraction -
\item Multiplication *
\item Division /
\item Equals =
\item Equal To ==
\item Not Equal To !=
\item Less Than <
\item Greater Than > 
\item Logical AND \&\&
\end{enumerate}

And there are more!
\par

Arithmetic operators include the mathematical symbols for addition (+), subtractions (-), division(/), multiplication(*) and equals(=).
These operators are used to directly manipulate and change variables.
For example we can add two variables together and then store that sum into a third variable.

\begin{lstlisting}[caption={Addition Operator}]
a = 10
b = 5
c = a + b

print c
\end{lstlisting}

We store the value of \pigVal{10} into the variable \pigVar{a} and the value \pigVal{5} into the variable \pigVar{b}.
Lastly, we store the sum of the variables \pigVar{a} and \pigVar{b} into the variable \pigVar{c}.
The output of this program will be \pigOut{15}.
\par

Comparison operators are used to compare variables against other variables or values.
The output of a comparison operator will be a boolean value (\pigVal{true} or \pigVal{false}).
Some of the popular comparison operators are the equals to (==), not equals to (!=), less than (<), less than equal to (<=), greater than (>) and greater than equal to (>=).

\begin{lstlisting}[caption={Comparison Operators}]
name = ``Brett''
a = 10
b = 5
c = a + b


if name == ``Brett'':
   print ``Name is Brett''

if a > 10:
   print ``A is greater than 10''

if b >= 5:
   print ``B is greater than or equal to 5''

if c != 6:
   print ``C is not 6''
\end{lstlisting}

In this example we are showing a few of the comparison operators.
The output of this code will be:\\*
\pigOut{Name is Brett\\*
B is greater than or equal to 5\\*
C is not 6}.

\subsection{Conclusion}

