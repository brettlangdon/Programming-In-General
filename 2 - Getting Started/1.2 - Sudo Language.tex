For the code examples presented in this resource I am going to be using a sudo language. The concept behind a sudo language
is to be able to present programming concepts in a language agnostic form so that the concepts can be translated to your
language of choice.
\newline
\\
So it is great that you have chosen lanauge X to use throughout this resource, but how is the sudo lanauge going to help you out?
Well, lets walk through a few examples and I will show you how the examples get translated to a few various languages.

\subsection{Example 1}
\begin{lstlisting}
name = "Brett"
if name == "Brett"
    print "Name Is Brett"
else
    print "Name Is Not Brett"
\end{lstlisting}

In this example we have a variable called "name" that we are assigning the value of "Brett". Then we are checking if the
variable name is equal to the value "Brett", if so then we want to print "Name Is Brett" otherwise if it does not then we want to 
print "Name Is Not Brett". As far as programming goes this is a fairly simple process but lets try and translate this example
to a few different languages to see how it is done.

PHP:
\begin{lstlisting}
<?php
$name = 'Brett';
if( $name === 'Brett' ){
    echo 'Name Is Brett';
} else{
    echo 'Name Is Not Brett';
}
\end{lstlisting}

C:
\begin{lstlisting}
int main{
    char* name = "Brett";
    if( name == "Brett" ){
        printf("Name Is Brett");
    } else{
        printf("Name Is Not Brett");
    }
    return 0;
}
\end{lstlisting}

Python:
\begin{lstlisting}
name = "Brett"
if name is "Brett":
    print "Name Is Brett"
else:
    print "Name Is Not Brett"
\end{lstlisting}

Node.JS:
\begin{lstlisting}
var name = "Brett";
if( name == "Brett" ){
    console.log("Name Is Brett");
} else{
    console.log("Name Is Not Brett");
}
\end{lstlisting}

Java:
\begin{lstlisting}
class Example1{
    public static void main( String[] args ){
        String name = "Brett";
        if( name.equals("Brett") ){
            System.out.println("Name Is Brett");
        } else{
            System.out.println("Name Is Not Brett");
        }
    }
}
\end{lstlisting}

Notice that all of the actual examples end up looking the same? That is the point of using the sudo language, so that we can
discuss the core concepts for each example. Also, notice the Python implementation, it is almost line for line, word for word
identical to the sudo language example.

\subsection{Example 2}
Since we have seen a fairly simple example above, lets take a look at a more complicated example. Do not be afraid if it does not make
too much sense right now, but try and notice the similarities between the sudo language and the actual code examples.

\begin{lstlisting}
class Person
    private name
    
    function getName()
        return this.name
    
    function setName( newName )
        this.name = newName


p = new Person()
p.setName("Brett")


if p.getName() == "Brett"
    print "Name Is Brett"
else
    print "Name Is Not Brett"
\end{lstlisting}

Ok, in this example we are using the concept of classes and objects. We are creating a new class called "Person" with a
private property "name" and two methods "getName" and "setName". "getName" will return the value of the private property "name" and
"setName" will take in a single parameter "newName" and set the private property "name"'s value to the value of "newName". Lastly we
are going to create a variable called "p" and have it be equal to a new instance of a "Person", set that instances name to "Brett" then like 
in the previous example we are going to get if the value of "p"'s private property "name" is equal to "Brett".
\newline
\\
Like I mentioned before, this example might go over the head of some people as it introduces some more advanced topics, but hopefully it
helps to understand how you can translate the sudo language.

PHP:
\begin{lstlisting}
<?php
class Person{
    private $name;
    
    public function getName(){
        return $this->name;
    }
    
    public function setname( $newName ){
        $this->name = $newName;
    }
}

$p = new Person();
$p->setName('Brett');

if( $p->getName() === 'Brett' ){
    echo 'Name Is Brett';
} else{
    echo 'Name Is Not Brett';
}
\end{lstlisting}

Java:
\begin{lstlisting}
class Person{
    private String name;
    
    public String getName(){
        return this.name;
    }
    
    public void setName( String newName ){
        this.name = newName;
    }
    
    public static void main(String[] args){
        Person p = new Person();
        p.setName("Brett");
        
        if( p.getName() == "Brett" ){
            System.out.println("Name Is Brett");
        } else{
            System.out.println("Name Is Not Brett");
        }
    }

}
\end{lstlisting}

Node.Js:
\begin{lstlisting}
var Person = function(){}
Person.prototype.getName = function(){
    return this.name;
}
Person.prototype.setName = function( newName ){
    this.name = newName;
}

var p = new Person();
p.setName("Brett");

if( p.getName() == "Brett" ){
    console.log("Name Is Brett");
} else{
    console.log("Name Is Not Brett");
}
\end{lstlisting}

Python:
\begin{lstlisting}
class Person:
    def getname( self ):
        return self.name
    def setName( self, newName ):
        self.name = newName

p = Person()
p.setName("Brett");

if p.getName() is "Brett":
    print "Name Is Brett"
else:
    print "Name Is Not Brett"
\end{lstlisting}

This example does a better job of showing how each language can tackle the concepts in a different manner but the core concepts
laid out by the sudo language can still be extrapolated and translated to each individual programming language. As long as the
language supports the concepts. As you may notice that I left out the implementation of C in this example. It is because C 
does not support the use of classes and objects, yes there are ways of completing this example in C using structs but that
is something that you should learn on your own.
\newline
\\

So now you have seen a few examples, hopefully enough to give you an idea of how the examples in this resource will be presented.
